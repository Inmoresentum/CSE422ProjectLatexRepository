
In the end, the use of machine learning techniques for spam email detection has proven to be an effective way to automatically filter out unwanted messages.
And to do so, we don't need to use advanced methods such as 10-level deep neural networks;
 instead, a simple Logistic Regression is way more than enough to detect 98\% of the time.
By training a model on a large dataset of labeled emails, we were able to achieve high levels of accuracy in predicting whether a given email was spam or not.
The ability to accurately identify and remove spam emails not only helps to protect individuals from potential scams and phishing attempts,
but also helps to reduce the overall amount of spam that is sent and received, making the internet a safer and more efficient place.
Overall, this project has demonstrated the power of machine learning in the fight against spam email.
\newpage
\singlespacing
\bibliographystyle{IEEEtran}
\bibliography{main}


%------ To create Appendix with additional stuff -------%
%\newpage
%\appendix
%\section{Appendix}
%Put data files, CAD drawings, additional sketches, etc.